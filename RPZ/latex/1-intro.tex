\Introduction

Физические тела, окружающие нас, обладают различными оптическими свойствами. Они, к примеру, могу отражать или пропускать световые лучи, также они могут отбрасывать тень. Эти и другие свойства нужно уметь наглядно показывать при помощи электронно-вычислительных машин. Этим и занимается компьютерная графика.

\textit{Компьютерная графика} -- представляет собой совокупность методов и способов преобразования информации в графическое представление при помощи ЭВМ. Без компьютерной графики не обходится ни одна современная программа. В течении нескольких десятилетий компьютерная графика прошла долгий путь, начиная с базовых алгоритмов, таких как вычерчивание линий и отрезков, до построения виртуальной реальности. 

Целью данной практики является разработка аналитической части проекта по дисциплине компьютерная графика на тему "визуализация маятника Ньютона".

В рамках выполнения работы необходимо решить следующие задачи:
\begin{enumerate}
	\item Описать предметную область работы. 
	\item Рассмотреть существующие алгоритмы построения реалистичных изображений. 
	\item Выбрать и обосновать выбор реализуемых методов или алгоритмов. 
	% 0\item Подробно изучить выбранный алгоритм.
	%\item Разработать программу на основе одного из существующих алгоритмов. Программа должна соответствовать техническому заданию и создавать реалистичное изображение. // Это уже не практика, а сама курсовая.
\end{enumerate}
