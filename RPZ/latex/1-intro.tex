\newpage
\chapter*{Введение}
\addcontentsline{toc}{chapter}{Введение}

Физические тела, окружающие нас, обладают различными оптическими свойствами. Они, к примеру, могу отражать или пропускать световые лучи,
также они могут отбрасывать тень. Эти и другие свойства нужно уметь наглядно показывать при помощи электронно-вычислительных машин.
Этим и занимается компьютерная графика.

\textit{Компьютерная графика} -- представляет собой совокупность методов и способов преобразования информации в графическое представление при помощи ЭВМ.
Без компьютерной графики не обходится ни одна современная программа. В течении нескольких десятилетий компьютерная графика прошла долгий путь, начиная с базовых
алгоритмов, таких как вычерчивание линий и отрезков, до построения виртуальной реальности.

Целью данного курсового проекта является разработка ПО по дисциплине компьютерная графика на тему "визуализация маятника Ньютона".

В рамках выполнения работы необходимо решить следующие задачи:

\begin{enumerate}
	\item описать предметную область работы;
	\item рассмотреть существующие алгоритмы построения реалистичных изображений;
	\item выбрать и обосновать выбор реализуемых методов или алгоритмов;
	\item подробно изучить выбранный алгоритм;
	\item разработать программу на основе одного из существующих алгоритмов;
	\item увеличить скорость работы выбранного алгоритма;
	\item выбрать и обосновать выбор языка программирования, для решения данной задачи.
\end{enumerate}
