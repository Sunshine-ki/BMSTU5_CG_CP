\newpage
\chapter*{Введение}
\addcontentsline{toc}{chapter}{Введение}

Физические тела, окружающие нас, обладают различными оптическими свойствами. Они, к примеру, могут отражать или пропускать световые лучи,
также они могут отбрасывать тень. Эти и другие свойства нужно уметь наглядно показывать при помощи электронно-вычислительных машин.
Этим и занимается компьютерная графика.

\textit{Компьютерная графика} представляет собой совокупность методов и способов преобразования информации в графическое представление при помощи ЭВМ.
Без компьютерной графики не обходится ни одна современная программа. В течении нескольких десятилетий компьютерная графика прошла долгий путь, начиная с базовых
алгоритмов, таких как вычерчивание линий и отрезков, до построения виртуальной реальности.

Целью данного курсового проекта является разработка ПО, которое
предоставляет визуализацию маятника Ньютона.

В рамках выполнения работы необходимо решить следующие задачи.

\begin{enumerate}
	\item Описать предметную область работы.
	\item Рассмотреть существующие алгоритмы построения реалистичных изображений.
	\item Выбрать и обосновать выбор реализуемых алгоритмов.
	\item Подробно изучить выбранные алгоритмы.
	\item Разработать программу на основе существующих алгоритмов.
	\item Снизить время работы выбранного алгоритма.
	\item Выбрать и обосновать выбор языка программирования, для решения данной задачи.
\end{enumerate}
