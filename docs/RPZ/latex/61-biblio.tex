\addcontentsline{toc}{chapter}{Список литературы}
\begin{thebibliography}{3}
	\bibitem{Vs}
	Visual Studio Code [Электронный ресурс], режим доступа: https://code.visualstudio.com/ (дата обращения: 02.10.2020)
	\bibitem{Win}
	Windows [Электронный ресурс], режим доступа:https://www.microsoft.com/ru-ru/windows/ (дата обращения: 02.10.2020)
	\bibitem{Lin}
	Linux [Электронный ресурс], режим доступа:https://www.linux.org.ru/ (дата обращения: 02.10.2020)
	\bibitem{Microsoft}
	Руководство по языку C\#[Электронный ресурс], - режим доступа: https://docs.microsoft.com/ru-ru/dotnet/csharp/ (дата обращения: 02.10.2020)
	\bibitem{Ubuntu}
	Ubuntu 18.04 [Электронный ресурс], режим доступа:https://releases.ubuntu.com/18.04/ (дата обращения: 02.10.2020)
	\bibitem{Stopwatch}
	Stopwatch Класс, [Электронный ресурс], режим доступа:
	https://docs.microsoft.com/ru-ru/dotnet/api/system.diagnostics.stopwatch?view=net-5.0  (дата обращения: 12.29.2020)
	\bibitem{tr1}
	Дымченко, Лев. Пример реализации в реальном времени метода
	трассировки лучей: необычные возможности и принцип работы. Оптимизация
	под SSE [Электронный ресурс],  режим доступа:https://www.ixbt.com/video/
	rt-raytracing.shtml. (дата обращения: 02.20.2020)
	\bibitem{tr2}
	Роджерс, Д. Алгоритмические основы машинной графики / Д. Роджерс. — Москва «Мир», 1989. — P. 512.
	\bibitem{tr3}
	Ю.М.Баяковский. Трассировка лучей из книги Джефа Проузиса [Электронный ресурс],
	режим доступа: ttps://www.graphicon.ru/oldgr/courses/cg99/notes/lect12/prouzis/raytrace.htm. (дата обращения: 02.20.2020)
\end{thebibliography}
